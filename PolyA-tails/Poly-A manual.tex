\documentclass[12pt]{article}
\usepackage[a2paper]{geometry}
\usepackage[russian]{babel}
\usepackage[utf8]{inputenc}
\usepackage{amssymb}
\usepackage{latexsym}
\usepackage{graphicx}
\usepackage{hyperref}

\begin{document}
\begin{center}
\begin{Large}
PolyAStatist - tool for collecting statistics about poly-A tails of non-LTR retrotransposones.
\end{Large}
\end{center}

Poly-A tail is important characteristic of Non-LTR retrotransposons. It could implicitly show age of transposition event and help with finding repeats de novo.
PolyAStatist works with output files of Repeat masker and outputs percentage of repeats with poly-A tails to total number of repeats. \\

\begin{Large}
Compilation: \\*
\end{Large}
To compile and run PolyAStatist you need g++ 4.8 or Microsoft Visual C++ compiler with с++11 support.  \\*

\begin{Large}
Usage: \\*
\end{Large}
To use PolyAStatist you should pass list of files to process. You could pass as many files as you want, but files should go in pares, where the first one is .fa.out file is fasta file and .fa.newout file from RepeatMasker output on that fasta file. \\*

\begin{Large}
Examples of correct usage: \\*
\end{Large}
user@ace:\~/cst\_v\_1\_1\$ ./PolyAStatist.exe chr1.fa.out chr1.fa chr2.fa.out chr2.fa chr3.fa.out  chr3.fa chr4.fa.out chr4.fa \\*
user@ace:\~/cst\_v\_1\_1\$ ./PolyAStatist.exe chrX.fa.out chrX.fa \\*

\begin{Large}
Examples of incorrect usage: \\*
\end{Large}
user@ace:\~/cst\_v\_1\_1\$ ./PolyAStatist.exe chr1.fa chr1.fa.out  chr2.fa chr2.fa.out chr3.fa chr3.fa.out  chr4.fa chr4.fa.out \\*
user@ace:\~/cst\_v\_1\_1\$ ./PolyAStatist.exe chr1.fa.out chr2.fa.out chr3.fa.out chr4.fa.out chr1.fa chr2.fa chr3.fa chr4.fa \\*

You could find full information about repeat masker usage at \href{http://www.repeatmasker.org/}{http://www.repeatmasker.org/}  \\*

But it is enough to run repeat masker like this: \\* 
RepeatMasker myFile.fasta \\*
Dont forget about species flag - it improves sensitivity a lot. The only importatant this is - PolyAStatistic works only with files, generated from single sequence. \\*

\begin{Large}
Workflow \\*
\end{Large}
To find polyA tails polyAStatistic make several steps: \\*
1) Filter LINE Retsrotransposones from another repeatitive elements. \\*
2) Merging some cases, when RepeatMasker incorrectly determine repeat with 5' inversion as pair of repeats. \\*
3) Cut off to short repeats (<1000 bp) as nonrepresentative. \\*
4) Check several positions on 3' end of repeat for poly-A tail \\*
5) Output the results \\*

To change some features (as length of repeat of filter another family of repeats) you could change the code as you wish - it has some comments for clarification. \\*

\begin{Large}
Output \\*
\end{Large}
PolyAStatist has following output: \\*
The first line has one of two variants: \\*
a) In ... found - if PolyAStatist worked with single pair of files \\*
b) In ...and x others found - if PolyAStatist worked with x + 1 pairs of files \\*
The second line is header od table: \\*
Family Of Repeat;Number of repeats;Number of repeats with poly-A tails;Percentage of repeat with poly-A tails \\*
Other lines contains rows of table in following format: \\*
L1P3;1440;672;0.4666\\*
L1PB3;888;672;0.7567\\*
L1HS;1560;1344;0.8615\\*
...\\*

For complaints and suggestions please write to 1dayac@gmail.com to Meleshko Dmitrii.\\*
you could find source code at \href{https://github.com/1dayac/AUSpringRepeats}{https://github.com/1dayac/AUSpringRepeats}
\end{document}